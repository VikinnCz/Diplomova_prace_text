\chapter{Výsledky studentské práce}



\section{Výběr vhodných metod pro extrakci vlastností hudební nahrávky} \label{sec:Exktrakce_vlastnosti_metody}

Vědeská komunita nabízí několik volně šířících knihoven obsahující techniky z oborů MIR. V této části práce jsou prozkoumány 3 knihovny zmíněné v bodě \ref{sec:Dostupna_reseni}. Jsou použity jejich funkce získání potřebných parametrů hudební nahrávky potřebných pro navazující diplomovou práci. Tyto funkce jsou mezi sebou porovnány z hlediska přesnosti výsledků, rychlosti výpočtů, jednoduchosti použití a možnosti využití pro komerční účely. 

\subsection{Detekce dob a tempa}



\subsection{Analýza chromavektorů}



\section{Návrh výsledného systému}

Návrh popisuje komplexní systém skládající se z několika částí, uživatelské rozhraní algoritmů pro získání pamrametrů hudební nahrávky a algoritmu generujícího SpectodaCode na základě získaných parametrů. V této kapitole je podrobně popsán návrh jednotlivých částí systému. 

\subsection{Uživatelské rozhraní}

Uživatelské rozhraní je reprezentováno webovou stránkou a je naprogramováno pomocí značkovacího jazyka \acs{HTML} spolu s formátováním v jazyce CSS. Funkčnosto webové stránky je zajištěna funkcemi jazyce JavaScript. Javascript také vytváří propojovací můstek pro komunikaci s vnistřním systémem v jazyce Python.

Jedná se o jednoduché webové rozhraní ve kterém uživatel nahraje hudební skladbu ve formátu .wav. Rozhranní obsahuje pole pro vložení cesty k hudební skladbě umožňující výběr ze souborů v uživatelově uložišti, 4 tlačítka pro výběr nálady
\uv{mood}. Jedná se o tlačítka: \uv{chill}, \uv{feeling happy}, \uv{hang out} a \uv{dancing}. Pod tlačítky pro výběr nálady se nachází tlačítko pro spuštění procesu generování SpectodaCodu.
 Poslední částí webového rozhraní je textové pole ve kterém se zobrazí vygenerovaný SpectodaCode.

% TODO Jednoduché blokové schéma postupu uživatele od otevření stránky k získání kódu.

Na blokovém schématu výše je zobrazen proces postupu uživatele skrze webové rozhraní. 

\subsection{Parametry hudební nahrávky} \label{sec:Parametry_nahravky}

Systém popsaný v bodě \ref{sec:System_generovani_animaci} vyžaduje vstupní data o hudební nahrávce. Tyto data jsou rozdělena 7 odlišných objektů. Každý z těchto objektů představuje určitou vlastnost analyzované nahrávky. Tyto vlastnosti jsou získány pomocí technik popsaných v bodě \ref{sec:Exktrakce_vlastnosti_metody}. Jednotlivé vlastnosti a jejich datové struktury jsou shrnuty v následujících bodech.

\begin{description}
    \item[Detekce dob] 
    \item[Tempo skladby]
    \item[Chromavektory]
    \item[Spektrální tok]
    \item[Periodicita]
    \item[Žánr]
    \item[Nálada]      
\end{description}

\subsection{Systém pro generování animací} \label{sec:System_generovani_animaci}

Vstupními parametry systému jsou získané parametry jež jsou popsány v bodě \ref{sec:Parametry_nahravky} 

% Možná lepší první popsat strukturu systému jak by se měly animace generovat a co k tomu budou potřeba za parametry. Následně teprve vytvořit sekci kde je popsán vzniklý kód pro analýzu jednotlivých parametrů. Také důležité jestli ukázaky kódu atd.. nebudou zahrnuty spíše v sekci Věběr vhodných metod pro extrakci vlastností hudební nahrávky. 