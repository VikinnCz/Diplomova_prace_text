\chapter*{Závěr}
\phantomsection
\addcontentsline{toc}{chapter}{Závěr}

V rámci diplomové práce byly splněny v úvodu stanovené cíle. Byla prozkoumána problematika oboru \acs{MIR} a na základě průzkumu navržena struktura systému pro generování spectoda kódu z parametrů hudební nahrávky. Systém byl následně realizován. Práce se zaobírá velkým množství oborů v oblasti \acs{MIR}, díky tomu je často problematika daných odvětví řešena pouze okrajově s cílem zachovat jednoduchost výsledné práce. V programu jsou z velké části implementovány již existující řešení v podobě volně dostupných knihoven pro nekomerční účely. Hlavní výstupem je výsledná logická struktura skládající bloky animací na základě získaných parametrů. 

První částí práce byl primárně teoretický průzkum a vytvoření blokové struktury znázorňující, jak by mohl proces generování probíhat. Zásadní byl průzkum použitelných parametrů hudební nahrávky a popis jakou roli dané parametry zaujmou v aplikaci. Výstupem první části byly blokové schémata a datová struktura systému popsaná v bodě \ref{sec:Navrh_systemu}. 

Druhým stanoveným cílem je porovnání metod pro extrakci parametrů. Zde byly porovnány primárně metody z knihoven Librosa, Madmom a Aubio. K hodnocení byla použita knihovna Mir\_eval. Výsledkem je osm Jupyter notebooků přiložených ve složce \textit{Methods\_comparisons}. Ne u všech parametrů bylo možné porovnat více metod k vůli jejich komplexnosti a náročnosti řešení. Například u klasifikace žánrů bylo vyzkoušeno pouze jedno řešeni. Podobně u segmentace hudební nahrávky poskytovala dobře dokumentované a funkční metody pouze knihovna Librosa. 

Realizace navrženého systému byla třetím cílem práce. Vzhledem k připraveným podkladům se jednalo primárně o přenesení návrhu do funkčního řešení. bylo nutné primárně vymyslet jednoduché propojení uživatelského rozhraní s vnitřní logikou systému. Propojení bylo nakonec vyřešeno pomocí frameworku Flask, díky tomu může aplikace pracovat na serveru a komunikovat s webovou stránkou pomocí \acs{HTTP} protokolu. Celý proces realizace je popsán v bodě \ref{sec:Realizace}. Vzniklé zdrojové kódy jsou přiložené v elektronické příloze \ref{sec:Obsah_prilohy}. 

V kapitole \ref{sec:Hodnoceni_systemu} jsou shrnuty problematiky vzniklého řešení a popsány návrhy na jeho zlepšení v rámci budoucího vývoje. Zkráceně generování spectoda kódu je funkční, ale animace nejsou vizuálně zajímavé. Jednou z příčin je nedostatečná databáze datasetů, která by potřebovala zapojení designera pro vytvoření vizuálně zajímavých datasetů animací. Animace jako takové dobře reagují na rytmickou skladbu nahrávky a systém dokáže komponovat bloky animací na základě struktury dané skladby. Pro další vývoj je potřebné stanovit parametry pro hodnocení vizuální stránky výsledných animací. Na základě stanovených parametrů provést ladění systému obnášející testování vlivu změn hodnot vah a parametrů logické struktury pro přiřazování bloků animací. 