\chapter{Ukázky zdrojových kódů}

\begin{minipage}{\linewidth}
	\begin{lstlisting}[
		breaklines = true,
		frame=single,
		numbers=left,
		caption={Zdrojový kód funkce \texttt{Find\_segment} v jazyce Python},
		label={code:Find_segment}]
		def Find_segment(y : list, beat_time):
		"""
		Function that finds segment from provided list in which is located the beat_time.

		Parameters
		----------
		y : list
			List of times where the segments boundaries are located
		beat_time : float
			Time of the beat which is wanted to locate.

		Returns
		----------
		start_segment : float
			Time where the located segment starts.
		end_segment : float
			Time where the located segment ends.
		"""

		y = np.asarray(y)
		idx = np.abs((y - beat_time)).argmin()

		if y[idx] > beat_time:
			idx -= 1

		start_segment = y[idx]
		try:
			end_segment = y[idx+1]
		except IndexError:
			end_segment = None

		return start_segment, end_segment

	\end{lstlisting}
\end{minipage}

\begin{minipage}{\linewidth}
	\begin{lstlisting}[
		breaklines = true,
		frame=single,
		numbers=left,
		caption={Zdrojový kód funkce \texttt{Find\_nearest\_beat} v jazyce Python},
		label={code:Find_nearest_beat}]
		def Find_nearest_beat(y : list, time):
		"""
		This function finds the nearest beat in given list.
	
		Parameters
		----------
		y : list
			List of times where the beats are located.
		time : float
			Time around that is searching for the nearest beat.
	
		Returns 
		----------
		idx : int
			Index of finded beat in the list.
		"""
		y = np.asarray(y)
		idx = np.abs((y - time)).argmin() 
		return idx

	\end{lstlisting}
\end{minipage}

\begin{minipage}{\linewidth}
	\begin{lstlisting}[
		breaklines = true,
		frame=single,
		numbers=left,
		caption={Zdrojový kód funkce \texttt{Calc\_strength} v jazyce Python},
		label={code:Calc_strength}]
		def __Calc_strength(self, onset_env):
		"""
		Calculate strength of beats.

		The function calculate beats strength based on onset envelope in time of the beat. Function also check range around the beat if there is some bigger value in onset envelope.

		Parameters
		----------
		onset_env : ndarray
			Onset envelope
		"""
		self.__strength = np.ones(len(self.__beats)) # Declaration of ones ndarray.
		i = 0

		for beat in self.__beats:
			try:
				index = np.where(self.__times == beat)[0] # Getting a timestamp of the beat.
				self.__strength[i] = self.__Max_of_range(int(index), onset_env) # Gets a biggest onset value in range around the timestamp of beat.
			except ValueError:
				self.__strength[i] = 0
			i += 1

		self.__strength = librosa.util.normalize(self.__strength) # The beat strength normalization between values 0-1.
	
	\end{lstlisting}
\end{minipage}

\begin{minipage}{\linewidth}
	\begin{lstlisting}[
		breaklines = true,
		frame=single,
		numbers=left,
		caption={Zdrojový kód funkce \texttt{Dataset\_selection} v jazyce Python},
		label={code:Dataset_selection}]
		def Dataset_selection(dataset_database : list[Dataset], genre_classification : GenreClassification, beat_tracking : BeatTracking, mood : int):

		# Get parameters 
		genre_predictions = genre_classification.genres_predictions
		tempo = beat_tracking.tempo
	
		genres_difs = []
	
		# Browsing thru all datasets
		for i, dataset in enumerate(dataset_database):
			genre_dif = 0
			d_genres_prediction = dataset.genre
			for key in d_genres_prediction:
				genre_dif += d_genres_prediction[key] - genre_predictions[key]
	
			genres_difs.append(np.abs(genre_dif))
		genre_pass_datasets = []
	
		# Get five datasets with smallest genre difference
		for i in range(5):
			index_of_min = int(np.argmin(genres_difs))
			genre_pass_datasets.append( dataset_database[index_of_min])
			genres_difs[index_of_min] = 255
	
		this_tempo_dif = 100
		selected_dataset = Dataset
		
		# Get dataset with same mood an smallest tempo diference
		for dataset in genre_pass_datasets:
			if dataset.mood == mood:
				new_tempo_dif = abs(dataset.tempo - tempo)
				if this_tempo_dif > new_tempo_dif:
					this_tempo_dif = new_tempo_dif
					selected_dataset = dataset
	
		return selected_dataset
	\end{lstlisting}
\end{minipage}

\chapter{Obsah elektronické přílohy} \label{sec:Obsah_prilohy}
Aplikace je psaná v programovacím jazyce Python ve verzi 3.11.6. Verze použitých knihoven jsou přiloženy v textovém dokumentu s názvem \textit{package\_versions.txt}. V průběhu obhajoby práce je aplikace dostupná na webové stránce \href{http://vikinn.pythonanywhere.com/}{zde}.

\bigskip
{\small
%
\dirtree{%.
.1 /\DTcomment{kořenový adresář přiloženého archivu}.
.2 Generator\_core\_structure\DTcomment{ zdrojové kódy aplikace}.
.3 static.
.4 styles.css.
.3 templates.
.4 main\_page.html.
.3 AnimationBlock.py.
.3 BeatTracking.py.
.3 Constants.py.
.3 Dataset.py.
.3 dataset\_database.json.
.3 GenreClassification.py.
.3 ChromaFeatures.py.
.3 Main.py.
.3 Segmentation.py.
.2 Matlab\_graphs\DTcomment{zdrojové kódy grafů}.
.3 ADSR.m.
.3 Discrete\_signal.m.
.3 Energy\_function.m.
.3 FFT.m.
.3 STFT.m.
.3 Spektrogram\_tok.
.4 Mel\_spektralni\_tok.m.
.4 Spektrogram.m.
.4 Spektralni\_tok.m.
.3 Waveform.m.
.2 Method\_comparisons\DTcomment{jupyter notebooky pro porovnání extrakce parametrů}.
.3 Beat\_tracking\_comparison.ipynb.
.3 Color\_palete.ipynb.
.3 Data\_preparation.ipynb.
.3 Dataset\_creating.ipynb.
.3 GenreClasification\_02.ipynb.
.3 Chroma\_vectors\_comparison.ipynb.
.3 Segmentation.ipynb.
.3 Signal\_rms.ipynb.
.2 package\_versions.txt\DTcomment{verze použitých balíčko pro python}.
}
}