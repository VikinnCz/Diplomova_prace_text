\chapter*{Úvod}
\phantomsection
\addcontentsline{toc}{chapter}{Úvod}

Diplomová práce se zabývá generováním světelných animací pro systém Spectoda, na základě analýze parametrů získaných z hudebních nahrávek. Spectoda je společnost specializující se na inteligentní řízení světel a světelných efektů. Aplikace vyvíjená v rámci této práce je primárně určena pro animace adresovatelných LED pásků. Práce je strukturovaná do třech cílů, které pokrývají celý proces vývoje -- od teoretického návrhu, přes průzkum a testování, až po vývoj finálního systému generujícího funkční kód pro ovládání LED pásků.

\begin{description}
    \item[Prvním cílem] -- je prozkoumat možnosti generování animací a identifikovat, které informace z nahrávek jsou pro tento proces nejrelevantnější. Na základě získaných dat navrhnout strukturu celého systému, tak aby byl snadno implementovatelný do webových aplikací a byl uživatelsky přívětivý -- generoval kód animace dostatečně rychle. Dalším výsledným parametrem je vizuální kvalita výsledné animace. 


    \item[Druhým cílem] -- je prozkoumat existující řešení v oblasti extrakce hudebních parametrů, známé pod oborem Music Information Retrieval (MIR), a vyhodnotit je z hlediska přesnosti a efektivity výpočtů. Z těchto poznatků budou vybrány nejvhodnější metody pro implementaci.
    \item[Třetím cílem] -- je naprogramovat a optimalizovat navržený systém. 
    
    Cílem práce je realizovat navržený systém a prozkoumat problematiku skutečného řešení a náročnosti docílit objektivně vizuálně pěkná animace. Výsledná struktura by měla představovat prototip na základě kterého bude následně postaven další vývoj k dosažení optiálních výsledků. 
    
\end{description}


% Starý Úvod
% -------------------------------------
% Práce se zabývá přípravou teoretických podkladů a průzkumem vhodných metod pro úspěšnou realizaci navazující diplomové práce v oblasti tvorby světelných animací na základě parametrizace hudební nahrávky. Získané znalosti jsou v semstrální práci využity pro výběr vhodných parametrů a návrh struktury výsledného systému. Tento systém bude generovat kód jež je následně převáděn na světelnou animaci pomocí řešení od společnosti Spectoda. Práce je rozdělena na 3 nasebe navazující cíle.


% \begin{description}
%     \item[Prvním cílem] je teoretický průzkum a vybudování dostatečné báze znalostí. Problematikou získávání informací z hudebních nahrávek se celosvětově zabývá vědní obor Music information retrieval zkráceně \acs{MIR}. V této oblasti je zapotřebí prozkoumat existující vhodná řešení a zjistit jakými postupy jsou realizovány. 
%     \item[Druhým cílem] je ze získaných znalostí navrhnout strukturu výsledného programu pro generování SpectodaCode, který je následně převáděn na světelné animace. Jsou prozkoumány a vybrány parametry hudebních děl potřebné pro realizaci systému. Při návrhu systému je nutné dbát aby byl realizovatelný, pro uživatele jednoduchý na použití a výpočetně nenáročný. Zároveň je kladen důraz na kvalitu generovaných animací.
%     \item[Třetím cílem] je praktická realizace získávání parametrů. Je nutné vyzkoušet dostupné metody pro analýzu vybraných parametrů, porovnat je mezi sebou na základě stanovených kritérií a určit jejich vhodnost pro použití ve struktuře výsledného systmu. Z průzkumu vzniknou návrhy pro zlepšení a úpravu vybraných parametrů. 
% \end{description}


% V rámci semestrální práce jsou popsány možnosti pro dolování parametrů z hudebních nahrávek a jejich analýzu.
% Tyto techniky jsou využity pro získání potřebných informací o skladbě.
% Například data o tempu a rozmístění dob, žánr a tónové či spektrální rozložení skladby.
% Dále je navržena struktura algoritmu sloužícího pro převod získaných parametrů na sekvence animací kompatibilních se systémem Spectoda.

% Práce je rozložena do tří na sebe navazujících cílů.
% Prvním z cílů je průzkum vědních oborů soustředících se na danou problematiku. Například \acs{MIR} (\acl{MIR}).
% Z existujícíh výzkumů jsou vybrány postupy analýzy hudebních signálů vyhovující pro použití v rámci výsledného algoritmu.

% Druhým cílem práce je navrhnout vnitřní strukuturu výsledného algoritmu převádějícího získané parametry na sekvence animací pro systém Spectoda.
% Důležitým úkolem je vymyslet jak bude docházet k takovému přenosu a co dané parametry ovlivní v rámci generování unikátních sekvencí animace.

% Poslední třetí cíl se zaobývá vytvořením funkčního systému pro získávání parametrů z hudební nahrávky.
% Důraz je kladen na využití dostupných moderních metod analýzy hudebních signálu.


% 3 cíle

% 1. Teoretický průzkum
% 2. Návrh struktruy výsledného systému
% 3. Výběr vhodných metod pro získání parametrů jejich realizace a porovnání
