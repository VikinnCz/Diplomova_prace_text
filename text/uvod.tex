\chapter*{Úvod}
\phantomsection
\addcontentsline{toc}{chapter}{Úvod}

V rámci semestrální práce vzniknou algoritmy analyzující hudební nahrávku.
Tyto algoritmy budou sloužit k získání potřebných dat jako jsou zejména beat detection,
získání tempa skladby a následné získání chromavektorů.
Při získávání parametrů je potřeba počítat s jejich následujícím využití v algoritmu generujícím animace pomocí systému Spectoda. 
\bigskip

V rámci semestrální práce jsou stanoveny 3 cíle.

Prvním z nich je nashromaždění dostatku teoretických informací o problematice \acs{MIR} (\acl{MIR})
a možnostech dolování informací z hudební nahrávky.

Druhým cílem práce pak je na základě získaných znalostí navrhnout vhodnou strukturu algoritmu pro generování sekvencí světelných animací pracujících na systému Spectoda.

Posledním cílem semestrální práce je právě vytvoření funkčního systému pro analýzu hudební nahrávky a dolování získávání předem stanovených parametrů.
