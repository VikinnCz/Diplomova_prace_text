\chapter*{Úvod}
\phantomsection
\addcontentsline{toc}{chapter}{Úvod}

V rámci semestrální práce jsou popsány možnosti pro dolování parametrů z hudebních nahrávek a jejich analýzu.
Tyto techniky jsou využity pro získání potřebných informací o skladbě.
Například data o tempu a rozmístění dob, žánr a tónové či spektrální rozložení skladby.
Dále je navržena struktura algoritmu sloužícího pro převod získaných parametrů na sekvence animací kompatibilních se systémem Spectoda.
%TODO: popsat více po dopsání celé práce.

Práce je rozložena do tří na sebe navazujících cílů.
Prvním z cílů je průzkum vědních oborů soustředících se na danou problematiku. Například \acs{MIR} (\acl{MIR}).
Z existujícíh výzkumů jsou vybrány postupy analýzy hudebních signálů vyhovující pro použití v rámci výsledného algoritmu.

Druhým cílem práce je navrhnout vnitřní strukuturu výsledného algoritmu převádějícího získané parametry na sekvence animací pro systém Spectoda.
Důležitým úkolem je vymyslet jak bude docházet k takovému přenosu a co dané parametry ovlivní v rámci generování unikátních sekvencí animace.

Poslední třetí cíl se zaobývá vytvořením funkčního systému pro získávání parametrů z hudební nahrávky.
Důraz je kladen na využití dostupných moderních metod analýzy hudebních signálu.
