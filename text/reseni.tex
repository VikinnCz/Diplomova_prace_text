\chapter{Teorie}

Semestrální práce se zejména zabývá problematikou \acs{MIR}. Popsánou v kapitole (!doplnit kapitolu!).
Důležitou roli zde hraje i úvaha nad realizací světelných animací.
Je důležité aby bylo přemýšleno nad principem reakce světelných animací na hudbu.
Nabízejí se otázky jak by měla daná animace reagovat na konkrétní děj skladby.
Jakým způsobem navrhnout strukturu \dots

V táto části je proto rozebírána teorie zpracování hudební nahrávky pomocí známých algoritmů jako je například \acs{FFT} (\acl{FFT}) 
či nabízené možnosti strojového učení. Struktura a možnosti systému Spectoda pro generování interaktivních světelných animací.
Uměleckou částí, jak by měla animace prezentovat hudbu.

\section{MIR - Music information retrieval}

\subsection{Beat and tempo detection}

\subsection{Parametrizace hudebních signálů}

\section{Systém Spectoda}

\section{Hudební signál jako animace}